\documentclass[a4paper,11pt]{article}
 \usepackage[utf8]{inputenc} %latin9 ou1
 \usepackage[T1]{fontenc}
 \usepackage[normalem]{ulem}
 \usepackage[french]{babel}
 \usepackage{verbatim}
 \usepackage{graphicx}
%\usepackage[letterpaper]{geometry}
%\geometry{verbose,tmargin=3cm,bmargin=3cm,lmargin=3cm}
%\usepackage{varioref}

%\usepackage{babel}
\usepackage{fancyhdr}
\lhead{\itshape Tutoriel CSO}
\cfoot{Page \thepage}
\pagestyle{fancy}

\usepackage{listings}   % need for code encapsulation
\lstset{
	language=C++,
	numbers=left, numberstyle=\tiny, stepnumber=1, numbersep=7pt,
	keywordstyle=\color{blue}\bfseries\emph,
	breaklines=true,
	frameround=fttt,
	basicstyle= \mdseries\scriptsize }
	
% Commandes personnelles
\def\clap#1{\hbox to 0pt{\hss #1\hss}} % Une commande sembleble  \rlap ou \llap, mais centrant son argument
\def\ligne#1{\hbox to \hsize{\vbox{\centering #1}}} % Une commande centrant son contenu ( utiliser en mode vertical)
% Une comande qui met son premier argument  gauche, le second au 
% milieu et le dernier  droite, la premire ligne ce chacune de ces
% trois boites coincidant
\def\haut#1#2#3{\hbox to \hsize{\rlap{\vtop{\raggedright #1}}\hss \clap{\vtop{\centering #2}} \hss \llap{\vtop{\raggedleft #3}}}}%
% Idem, mais cette fois-ci, c'est la dernire ligne
\def\bas#1#2#3{\hbox to \hsize{\rlap{\vbox{\raggedright #1}} \hss \clap{\vbox{\centering #2}} \hss \llap{\vbox{\raggedleft #3}}}}%
    
    
% La commande \maketitle
\makeatletter
	\def\maketitle{%
	  \thispagestyle{empty}\vbox to \vsize{%
		\vspace{5mm} \ligne{\Huge \@title}
		\vspace{1cm} \haut{Supervisé par \@supervisor}{}{\@follower}
		\vspace{3mm}\hrule
		\vfill
		\bas{}{\@location, \@date}{}
		}%
	  \cleardoublepage
	  }

	% Les commandes permettant de dfinir la date, le lieu, etc.
	\def\date#1{\def\@date{#1}}
	\def\title#1{\def\@title{#1}}
	\def\location#1{\def\@location{#1}}
	\def\blurb#1{\def\@blurb{#1}}
	\def\supervisor#1{\def\@supervisor{#1}}
	\def\follower#1{\def\@follower{#1}}
	\def\email#1{\def\@email{\small{#1}}}
	% Valeurs par dfaut
	\date{\today}
\makeatother


  \title{Tutoriel CSO}
  \email{jerome.gazel@centraliens-nantes.net}
  \location{Nantes}

  \supervisor{M.\ Olivier Roux}
  \follower{\'{E}cole Centrale de Nantes}


\begin{document}
\maketitle

\section{Installer CSO}

\begin{enumerate}
\item 
  \begin{verbatim} 
  $ sudo aptitude install scala
  \end{verbatim}
\item Créer un répertoire dans lequel vous travaillerez : 
  \begin{verbatim} 
  $ mkdir projet-cso
  \end{verbatim}
\item Télécharger les fichiers requis (adapter la version si nécessaire)
  \begin{verbatim} 
  $ cd projet-cso
  $ wget http://www.scala-lang.org/downloads/distrib/files/scala-2.8.1.final.tgz
  $ wget http://users.comlab.ox.ac.uk/bernard.sufrin/CSO/cso-sources-scala2.8.0.tgz
  \end{verbatim}
  
\item Décompresser les archives scala-2.8.1.final.tgz et cso-sources-scala2.8.0.tgz
  \begin{verbatim} 
  $ tar -xvf scala-2.8.1.final.tgz
  $ tar -xvf cso-sources-scala2.8.0.tgz
  $ rm cso-sources-scala2.8.0.tgz
  $ rm scala-2.8.1.final.tgz
  $ mv scala-2.8.1.final Scala
  \end{verbatim}

\item Créer un répertoire ScalaLocal et déplacez-y le xml suivant
  \begin{verbatim} 
  $ mkdir ScalaLocal
  $ mv scalatasks.xml ScalaLocal/
  \end{verbatim}

\item Modifier le build pour la compilation
  \begin{verbatim} 
  $ gedit build.xml
  \end{verbatim}
  Et modifier les lignes 11 et 12 par les \$\{user.home\}/ par \$\{user.home\}/projet-cso/; en prenant en compte le chemin d'accès dans lequel vous avez créé votre fichier.
  
\item Compiler via la commande
  \begin{verbatim} 
  $ ant jar
  \end{verbatim}  
\end{enumerate}


\end{document}

