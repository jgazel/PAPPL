\documentclass[a4paper,11pt]{article}
 \usepackage[utf8]{inputenc} %latin9 ou1
 \usepackage[T1]{fontenc}
 \usepackage[normalem]{ulem}
 \usepackage[french]{babel}
 \usepackage{verbatim}
 \usepackage{graphicx}
%\usepackage[letterpaper]{geometry}
%\geometry{verbose,tmargin=3cm,bmargin=3cm,lmargin=3cm}
%\usepackage{varioref}

%\usepackage{babel}
\usepackage{fancyhdr}
\lhead{\itshape Compte-Rendu d'Avancement 2}
\cfoot{Page \thepage}
\pagestyle{fancy}

\usepackage{listings}   % need for code encapsulation
\lstset{
	language=C++,
	numbers=left, numberstyle=\tiny, stepnumber=1, numbersep=7pt,
	keywordstyle=\color{blue}\bfseries\emph,
	breaklines=true,
	frameround=fttt,
	basicstyle= \mdseries\scriptsize }
	
% Commandes personnelles
\def\clap#1{\hbox to 0pt{\hss #1\hss}} % Une commande sembleble  \rlap ou \llap, mais centrant son argument
\def\ligne#1{\hbox to \hsize{\vbox{\centering #1}}} % Une commande centrant son contenu ( utiliser en mode vertical)
% Une comande qui met son premier argument  gauche, le second au 
% milieu et le dernier  droite, la premire ligne ce chacune de ces
% trois boites coincidant
\def\haut#1#2#3{\hbox to \hsize{\rlap{\vtop{\raggedright #1}}\hss \clap{\vtop{\centering #2}} \hss \llap{\vtop{\raggedleft #3}}}}%
% Idem, mais cette fois-ci, c'est la dernire ligne
\def\bas#1#2#3{\hbox to \hsize{\rlap{\vbox{\raggedright #1}} \hss \clap{\vbox{\centering #2}} \hss \llap{\vbox{\raggedleft #3}}}}%
    
    
% La commande \maketitle
\makeatletter
	\def\maketitle{%
	  \thispagestyle{empty}\vbox to \vsize{%
		\vspace{5mm} \ligne{\Huge \@title}
		\vspace{1cm} \haut{Supervisé par \@supervisor}{}{\@follower}
		\vspace{3mm}\hrule
		\vfill
		\bas{}{\@location, \@date}{}
		}%
	  \cleardoublepage
	  }

	% Les commandes permettant de dfinir la date, le lieu, etc.
	\def\date#1{\def\@date{#1}}
	\def\title#1{\def\@title{#1}}
	\def\location#1{\def\@location{#1}}
	\def\blurb#1{\def\@blurb{#1}}
	\def\supervisor#1{\def\@supervisor{#1}}
	\def\follower#1{\def\@follower{#1}}
	\def\email#1{\def\@email{\small{#1}}}
	% Valeurs par dfaut
	\date{\today}
\makeatother


  \title{Compte-Rendu d'Avancement 2}
  \email{jerome.gazel@centraliens-nantes.net}
  \location{Nantes}

  \supervisor{M.\ Olivier Roux}
  \follower{\'{E}cole Centrale de Nantes}


\begin{document}
\maketitle

\section{Répartition des tâches effectuées}

Après une semaine depuis le dernier compte-rendu d'avancement, nous avions quelque difficultés à utiliser CSO : après avoir écrit un tutoriel sur son installation, nous avons essayé de lancer lyfe, un des programmes fournis avec CSO pour comprendre l'implémentation de la programmation concurrente.
Cependant, nous ne réussissions pas à compiler le programme, puis après avoir résolu ce premier problème, nous n'arrivions pas à executer le programme. Nous avons donc décidé de se répartir les tâches afin de continuer d'avancer dans le projet d'application : Jérôme aura en charge de continuer à étudier CSO et Clément s'occuperait de l'installation de JCSP et de sa prise en main.

\section{\'{E}tude de CSO}

Jérôme a continué la deuxième semaine les recherches sur CSO. Après de nombreuses heures de travail et de recherche, il a constaté que le tutoriel que nous avions rédigé lors du dernier rapport d'avancement ne convenait pas. Il l'a donc repris puis a enfin réussi à exécuter le programme lyfe. Il a rédigé également un HelloWorld à l'aide de CSO. Pour le prochain rapport d'avancement, Jérôme cherchera à maîtriser encore plus CSO, et commencer à implémenter un des problèmes classiques, conformément au cahier des charges du projet d'application.

\section{\'{E}tude de JCSP}

Après avoir tenté de faire fonctionner Lyfe sur CSO, Clément a proposé de scinder le groupe afin de pouvoir avancer dans le projet. Il a installé JCSP et étudié les différents exemples fournis. Il a commencé à prendre en main cette bibliotèque et a commencé à implémenter le problème des philosophes. Un problème rencontré est celui des classpaths. EN En effet, pour le moment, afin de fonctionner, les bibliotèques doivent à chaque fois être copiées dans le répertoire des exemples et décompressées dans celui-ci. L'un des objectifs pour le prochain rapport d'avancement sera d'avoir résolu ce problème et d'avoir terminé l'implémentation du dîner des philosophes. Si le temps est suffisant, les objectifs secondaires seront de réfléchir à l'implémentation d'un des autres programmes classiques ou de commencer la rédaction du tutoriel.

\end{document}

