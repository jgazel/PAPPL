\documentclass[a4paper,11pt,french]{report}
 \usepackage[utf8]{inputenc} %latin9 ou1
 \usepackage[T1]{fontenc}
 \usepackage[normalem]{ulem}
 \usepackage[letterpaper]{geometry}
 \usepackage{lmodern} %fonte latin modern
 \usepackage[francais]{babel}
 \usepackage{verbatim}
 \usepackage{graphicx}
 \usepackage{multicol}
 \usepackage{hyperref}
 \hypersetup{colorlinks=true,linkcolor=blue,urlcolor=red}
 \usepackage{varioref}
%\usepackage[letterpaper]{geometry}
%\geometry{verbose,tmargin=3cm,bmargin=3cm,lmargin=3cm}
%\usepackage{varioref}

%\usepackage{babel}
\usepackage{fancyhdr}
\cfoot{Page \thepage}
\pagestyle{fancy}

\usepackage{listings}   % need for code encapsulation
\lstset{
	language=PHP,
	numbers=left, numberstyle=\tiny, stepnumber=1, numbersep=7pt,
	keywordstyle=\bfseries\emph,
	breaklines=true,
	frameround=fttt,
	basicstyle= \mdseries\scriptsize }
\lstset{language=bash,basicstyle=\scriptsize,commentstyle=\small\itshape,stringstyle=\ttfamily,numbers=left,numberstyle=\tiny,stepnumber=1,showstringspaces=false}
	
% Commandes personnelles
\def\clap#1{\hbox to 0pt{\hss #1\hss}} % Une commande sembleble  \rlap ou \llap, mais centrant son argument
\def\ligne#1{\hbox to \hsize{\vbox{\centering #1}}} % Une commande centrant son contenu ( utiliser en mode vertical)
% Une comande qui met son premier argument  gauche, le second au 
% milieu et le dernier  droite, la premire ligne ce chacune de ces
% trois boites coincidant
\def\haut#1#2#3{\hbox to \hsize{\rlap{\vtop{\raggedright #1}}\hss \clap{\vtop{\centering #2}} \hss \llap{\vtop{\raggedleft #3}}}}%
% Idem, mais cette fois-ci, c'est la dernire ligne
\def\bas#1#2#3{\hbox to \hsize{\rlap{\vbox{\raggedright #1}} \hss \clap{\vbox{\centering #2}} \hss \llap{\vbox{\raggedleft #3}}}}%
    
    
% La commande \maketitle
\makeatletter
	\def\maketitle{%
	  \thispagestyle{empty}\vbox to \vsize{%
		\vspace{3cm} \ligne{\Huge \textsc \@title}
		\vspace{7mm} \ligne{\Large \@subtitle}
		\vspace{1cm} \haut{Supervisé par \@supervisor}{}{\@follower}
		\vspace{3mm}\hrule
		\vfill
		\haut{}{Jérôme \textsc{Gazel}}{}
		\haut{}{$\&$}{}
		\haut{}{Clément \textsc{Schiano de Colella}}{}
		\vspace{3cm}
		\bas{}{\@location, \@date}{}
		}%
	  %\cleardoublepage
	  }

	% Les commandes permettant de dfinir la date, le lieu, etc.
	\def\date#1{\def\@date{#1}}
	\def\title#1{\def\@title{#1}}
	\def\subtitle#1{\def\@subtitle{#1}}
	\def\location#1{\def\@location{#1}}
	\def\blurb#1{\def\@blurb{#1}}
	\def\supervisor#1{\def\@supervisor{#1}}
	\def\follower#1{\def\@follower{#1}}
	\def\email#1{\def\@email{\small{#1}}}
	% Valeurs par dfaut
	\date{\today}
\makeatother


  \title{Programmation Concurrentielle}
  \subtitle{Communicating Scala Objects\\Java Communicating Sequential Processes}
  \location{Nantes}
  \lhead{\itshape PAPPLE - 2011}
  \supervisor{M.\ Olivier \textsc{Roux}}
  \follower{\'Ecole Centrale de Nantes}

\begin{document}
\maketitle
\tableofcontents

\part{Introduction}

L'importance de la programmation concurrente n'a cessé de croître depuis quelques années. Sa connaissance et sa maîtrise deviendra à terme une nécessité pour tous les développeurs au regard du besoin de parallélisation des tâches. Pour mettre en place, la programmation concurrente, il existe déjà CSP. Imaginé par Hoare, cet algèbre de processus permet de modéliser les différentes interactions entre systèmes. Ce langage a connu plusieurs implémentations pour les principaux langages de programmation et notamment pour Java et Scala.
Notre travail a donc consisté à prendre en main ces bibliotèques, voire le langage pour Scala, puisque nous  connaissions peu ce langage avant ce projet.
Nous exposerons donc leur mise en place. Les principales caractéristiques de ces implémentations seront décrites, puis nous exposerons les programmes classiques de la programmation concurrente décrits à l'aide de ces implémentations. Enfin, nous dresserons un bilan de notre travail.

\part{Implémentation des différents langages}
\chapter[CSO]{Communicating Scala Objects -- CSO}

\section{Problème sur CSO}

Sous la supervision de notre encadrant M.\ \textsc{Roux}, nous nous sommes penchés ces dernières semaines sur CSO -- Communcating Scala Objects. Ce language est une extension de Scala, et permet d'implémenter tout algorithme de programmation concurrentielle.\\

Après avoir construit un fichier \textsf{buid.xml} capable de compiler, nettoyer, lancer le programme en adjoignant les bonnes ressources via une simple commande, puis avoir saisi la syntaxe nécessaire d'abord à Scala, puis à CSO, les programmes 'producteur-consommateur' et 'lecteur-écrivain'étaient l'un des objectifs de notre projet.\\

Cependant, après des résultats infructueux, il semblerait que notre équipe ait mis à jour un problème relevant du comportement de CSO vis-à-vis des processus concurrents. En effet, leur ordre d'arrivée dépend de leur position dans la sélective; ce qui ne devrait pas être le cas.\\ 

En d'autres termes, les processus ne s'executent que si l'un d'entre eux se finit (à la manière séquentielle d'un \textsc{Fifo}). Voici quelques exemples qui nous ont amenés à cette conclusion.

\subsection{Illustration du problème}

Pour illustrer ces propos, considérons le programme suivant \textsf{HelloWorld} qui lancent plusieurs procesus en parallèle, chacun ayant soit la tâche d'afficher "Hello" soit d'afficher "World".\\
La syntaxe-clef ici est la double-barre ||, qui permet de décrire une sélective, mais donne aussi la possiblité via l'opérateur \textsf{for} de lancer \textit{n} processus en parallèle.\\
Cette syntaxe est notamment décrite dans le rapport émis par Dr.\ \textsc{Suffrin}\footnote{Page personnelle de Bernard \textsc{Suffrin} à l'adresse \url{http://users.comlab.ox.ac.uk/bernard.sufrin/}} \cite{cpa2008-cso}, ainsi que dans le programme Lyfe qu'il met à disposition sur son site internet.
\medskip

\begin{lstlisting}[frame=trBL]
package helloworld
import ox.CSO._

object Hello_Par {

    val NP = 3
    val NC = 4
    
    def main( args: Array[String] ) {
  
        println("Debut du programme HelloWord Par :")
        ( 
           || ( for (i <- 0 until NP) yield Hello(i) )
        || 
           || ( for (i <- 0 until NC) yield World(i) )
        )()
        
    }


    def Hello( i: Int ) : PROC={
    	say("Hello" + i)
    } 

    def World( i: Int ) : PROC={
    	say("World" + i)
    }

    def say( word: String ) {
      println(word)
    }  
}
\end{lstlisting}

Les résultats obtenus avec cette syntaxe sont présentés dans la figure~\vref{fig:1HW}.

\begin{figure}[h]
\centering
\begin{multicols}{2}
\begin{lstlisting}[frame=trBL]
     [exec] Debut :
     [exec] Hello0
     [exec] Hello1
     [exec] World3
     [exec] World2
     [exec] World0
     [exec] World1
     [exec] Hello2
     [exec] Debut :
     [exec] Hello2
     [exec] Hello1
     [exec] Hello0
     [exec] World3
     [exec] World2
     [exec] World0
     [exec] World1
     [exec] Debut  :
     [exec] Hello2
     [exec] Hello0
     [exec] Hello1
     [exec] World3
     [exec] World0
     [exec] World2
     [exec] World1
     [exec] Debut  :
     [exec] Hello2
     [exec] Hello0
     [exec] Hello1
     [exec] World3
     [exec] World2
     [exec] World1
     [exec] World0
     [exec] Debut  :
     [exec] Hello2
     [exec] Hello1
     [exec] Hello0
     [exec] World3
     [exec] World2
     [exec] World0
     [exec] World1
     [exec] Debut  :
     [exec] Hello2
     [exec] Hello1
     [exec] Hello0
     [exec] World3
     [exec] World2
     [exec] World0
     [exec] World1
     [exec] Debut  :
     [exec] Hello2
     [exec] Hello0
     [exec] Hello1
     [exec] World3
     [exec] World1
     [exec] World0
     [exec] World2
     [exec] Debut  :
     [exec] Hello2
     [exec] Hello0
     [exec] Hello1
     [exec] World3
     [exec] World2
     [exec] World1
     [exec] World0
     [exec] Debut  :
     [exec] Hello2
     [exec] World2
     [exec] Hello1
     [exec] Hello0
     [exec] World3
     [exec] World0
     [exec] World1
     [exec] Debut  :
     [exec] Hello2
     [exec] Hello0
     [exec] Hello1
     [exec] World3
     [exec] World2
     [exec] World0
     [exec] World1
     [exec] Debut  :
     [exec] Hello2
     [exec] Hello1
     [exec] Hello0
     [exec] World3
     [exec] World2
     [exec] World0
     [exec] World1
     [exec] Debut  :
     [exec] Hello2
     [exec] Hello1
     [exec] Hello0
     [exec] World3
     [exec] World2
     [exec] World0
     [exec] World1
     [exec] Debut  :
     [exec] Hello2
     [exec] Hello0
     [exec] Hello1
     [exec] World3
     [exec] World2
     [exec] World0
     [exec] World1
\end{lstlisting}
\end{multicols}
\caption{Résultat du premier HelloWorld en programmation concurrente}
\label{fig:1HW}
\end{figure}

On remarque bien que les processus dépendent de leur place dans la sélective, et leur ordre est même (sauf exceptions) quasiment le même.

\subsection{Pour aller plus loin}

Ce problème peut certes paraître minime. L'essentiel est bien sûr que l'ensemble des processus se comporte comme s'il tournait en parallèle.\\ Ici tout porte à croire que c'est cependant le cas. Mais un autre exemple démontre la séquencialité de l'exécution (\emph{cf.} l'encadré suivant)

\begin{lstlisting}[frame=trBL, title={Contre-exemple démontrant la séquentialité de l'exécution}]
package gazel_schiano_cso
import ox.CSO._

object ProdCons {
    val NP = 4
    val NC = 1
    val N = 4
    var enVie = true

    val left, mon, right = ManyMany[String]

    def main( args: Array[String] ) {
        
        println("Debut du programme Producteur-Consommateur")
        ( 
           || ( for (i <- 0 until NP) yield Producteur(i, mon) )
        || 
           || ( for (i <- 0 until NC) yield Consommateur(i, mon) )
        || 
           Buffer(mon) 
        )()
    }
\end{lstlisting}
\begin{lstlisting}[frame=trBL, title={Contre-exemple, suite : Producteur}, firstnumber=last]    
    
    def Producteur(i : Int, out : ![String]) : PROC = {
        var nb = 0
        var message = ""
        println("Creation d'un producteur (" + i + ").")
        
        //repeat(enVie) {
            println("   Boucle Producteur -- Objet numero " + nb)
            sleep(i)
            message = "Objet numero " + nb
            println(message)
            mon ! message
            out ! message
            println("Production " + i + " : " + message)
            nb = nb + 1
        //}
    }
\end{lstlisting}
\begin{lstlisting}[frame=trBL, title={Contre-exemple, suite : Consommateur et Buffer}, firstnumber=last]
    def Consommateur(i : Int, in : ?[String]) : PROC = {
        var k = 0
        println("Creation d'un consommateur (" + i + ") :: enVie? " + enVie)
        
        //repeat(enVie) {
            println("   Boucle Consommateur")
            in ? { message => 
                    {
                        mon ! message 
                        right! message 
                        println("Consommateur " + i + " : " + message)
                    } 
                }
            //sleep(500*i)
        //}
    }  
    

    def  Buffer( tuyau : ?[String] ) : PROC = {
        println("Entree Buffer")
        tuyau ? { v => {println(v)}}
    }
    
}
\end{lstlisting}

Ce qui donne ici à l'execution l'encadré \vref{fig:contreexemple-resultat}, qui se fige lors de l'attente de l'objet via le pipe.

\begin{figure}[h]
\begin{lstlisting}[frame=trBL]
     [exec] Debut du programme Producteur-Consommateur
     [exec] Creation d'un producteur (0).
     [exec]    Boucle Producteur -- Objet numero 0
     [exec] Objet numero 0
     [exec] Production 0 : Objet numero 0
     [exec] Creation d'un producteur (1).
     [exec]    Boucle Producteur -- Objet numero 0
     [exec] Objet numero 0
     [exec] Production 1 : Objet numero 0
     [exec] Creation d'un producteur (2).
     [exec]    Boucle Producteur -- Objet numero 0
     [exec] Objet numero 0
     [exec] Production 2 : Objet numero 0
     [exec] Creation d'un producteur (3).
     [exec]    Boucle Producteur -- Objet numero 0
     [exec] Objet numero 0
     [exec] Production 3 : Objet numero 0
     [exec] Creation d'un consommateur (0) :: enVie? true
     [exec]    Boucle Consommateur
\end{lstlisting}
\caption{Contre-exemple démontrant la séquentialité de l'exécution [Résultats]}
\label{fig:contreexemple-resultat}
\end{figure}

\paragraph{Constats} L'encha\^inement de l'instanciation se fait de manière séquentielle; d'abord en commençant par les porducteurs, puis par le consommateur.

\paragraph{Que se passe-t-il ?} Il est clair que l'instanciation se fait de manière séquentielle, et que le Buffer \textit{mon} n'exite pas encore. De ce fait, l'essai de l'envoi via le pipe vers \textit{mon} ne peut pas aboutir, et on aboutir à un interblocage...

\section{Résolution du problème}
\subsection{Instanciation}

Nous l'avons vu, le problème se situe au niveau de l'instanciation. Nous sommes passés à c\^oté d'un point important dans le langage Scala: la différenciation entre les \textsf{object} et les \textsf{class}.\\

Essayons d'écrire un classe, et de définir à l'intérieur une méthode \textsf{proc}:

\begin{lstlisting}[frame=trBL]
package helloworld
import ox.CSO._
import ox.Format._
import ox.cso.Components.{console}

object Hello_Pipe {

    var enVie = true

    val pipe = OneOne[String]

    def main( args: Array[String] ) {
        println("Debut du programme Hello_Pipe")
        var sender = new Sender(pipe)
        var receiver = new Receiver(pipe)
        ( 
           sender.process
        || 
           receiver.process
        )()
    }

    class Sender(out: ![String])
    {
        val process : PROC =
        { 
            println("Creation d'un sender")
            repeat 
            {
                var message = "Mon Objet"
                println(">> Envoi de \"" + message + "\"...")
                out!message
                sleep(500)
            }
        }  
    }

    class Receiver(in: ?[String])
    {
        val process : PROC =
        { 
            println("Creation d'un receiver")
            var message =""
            repeat 
            {
                message=in ?; 
                println("\"" + message + "\" recu.")
            }
         }  
    }   
}
\end{lstlisting}
\bigskip

Ce problème s'apparente à un algorithme de producteur-consommateur, car la production crée le message, tandis que la consommation l'affiche, tout cela via un pipe.\\
Le résultat obtenu est le suivant:


\begin{lstlisting}[frame=trBL]
Buildfile: /home/jerome/Documents/Pappl/git/CSO/HelloWorld/build.xml

pipe:
     [exec] Debut du programme Hello_Pipe
     [exec] Creation d'un sender
     [exec] Creation d'un receiver
     [exec] Envoi de "Mon Objet"... >>
     [exec] <<Mon Objet recu
     [exec] Envoi de "Mon Objet"... >>
     [exec] <<Mon Objet recu
     [exec] Envoi de "Mon Objet"... >>
     [exec] <<Mon Objet recu
     [exec] Envoi de "Mon Objet"... >>
     [exec] <<Mon Objet recu
     [exec] Envoi de "Mon Objet"... >>
     [exec] <<Mon Objet recu
     [exec] Envoi de "Mon Objet"... >>
     [exec] <<Mon Objet recu
     [exec] Envoi de "Mon Objet"... >>
     [exec] <<Mon Objet recu
     [exec] Envoi de "Mon Objet"... >>
     [exec] <<Mon Objet recu
     [exec] Envoi de "Mon Objet"... >>
     [exec] <<Mon Objet recu
     [exec] Envoi de "Mon Objet"... >>
     [exec] <<Mon Objet recu
     [exec] Envoi de "Mon Objet"... >>
     [exec] <<Mon Objet recu
     [exec] Envoi de "Mon Objet"... >>
     [exec] <<Mon Objet recu
\end{lstlisting}

\bigskip

\textbf{Constats} L'envoi d'un message par le \textsf{Sender} se traduit par une immédiate impression du message par le \textsf{Receiver}. Le programme tourne comme prévu.

\paragraph{Remarque} Nous pouvons introduire ici la notion de rendez-vous avec la classe Receiver; en effet, il est possible de sustituer aux deux lignes contenues dans le \textsf{repeat\{\}} la ligne suivante
\begin{center}
\begin{verbatim}
in ? { message => { println("<<Reception : " + message) }  }
\end{verbatim}
\end{center}
où la récpetion d'un message dans le pipe exécutera la fonction contenue entre crochets


\paragraph{Que s'est-il passé ?} L'astuce se situe dans le \textsf{main}. Le fait d'avoir des classes nous permet une instanciation préléminaire sans lancer l'objet.\\
La sélective lançant le processus semble alors bien s'exécuter en parallèle.\\

\subsection{Un dernier effort}
Certes le programme précédent nous donne un résultat satisfaisant; mais il reste le problème du nombre. En effet, nous n'avons considérés ici qu'un couple de sender-receiver, relié par un seul pipe.\\
Nous avons ommis de parler en outre des différents pipes mis à disposition par le langage CSO, et de leur utilisation dans les multiples situations.

\subsubsection{Les différents pipes}
Nous pouvons lire dans le rapport du Docteur~\textsc{Suffrin} \cite{cpa2008-cso}, page 4 :
\begin{itemize}
\renewcommand{\labelitemi}{$\diamond$}
\item \textsf{OneOne[T]} -- Un seul et unique process peut accéder à la fois au port de sortie et d'entrée.
\item \textsf{ManyOne[T]} -- Un seul et unique process peut accéder au port d'entrée. Les process qui souhaitent accéder à sa sortie en ont la possibilité dans un ordre aléatoire.
\item \textsf{OneMany[T]} -- Un seul et unique process peut accéder au port de sortie. Les process qui souhaitent accéder à son entrée en ont la possibilité dans un ordre aléatoire.
\item \textsf{ManyMany[T]} -- Autant de process peuvent accéder à la fois au port de sortie et d'entrée. L'écriture et la lecture se font dans un ordre aléatoire.
\end{itemize}

\section{Algorithmes classiques}
\subsection{Producteur-Consommateur}

L'objet de cette sous-section est de présenter l'implémentation de l'algorithme classique de producteur-consommateur en CSO.\\
Pour la communication des différents processus, nous utiliserons deux de canaux (\textsf{val left,right = ManyMany[String]}) qui permettront de faire le lien entre les producteurs et le buffer d'une part, puis le buffer et les consommateurs d'autre part.

\begin{lstlisting}[frame=trBL,title={Producteurs-Consommateurs avec un buffer : En-tête}]
package gazel_schiano_cso
import ox.CSO._

object ProdCons {
    val NP = 4
    val NC = 1
    
    val left,right = ManyMany[String]

    def main( args: Array[String] ) {
        
        println("Debut du programme Producteur-Consommateur (" + NP + "," + NC + ").")
        
        // Instanciation
        val prods =  for (i <- 0 until NP) yield new Producteur(i, left)
        val cons = for (i <- 0 until NC) yield new Consommateur(i, right)
        val tampon = new Buffer(left,right)
        
        // Lancement en parallele des processus
        ( 
           || ( for (i <- 0 until NP) yield prods(i).produire )
        || 
           || ( for (i <- 0 until NC) yield cons(i).consommer )
        || 
           tampon.process 
        )()
    }
    
\end{lstlisting}

\begin{lstlisting}[frame=trBL,title={Producteurs-Consommateurs : Producteur}, firstnumber=last]
    class Producteur(i: Int, out: ![String]) {
        def produire : PROC = {
            var nb = 0
            var message = ""
            println("Creation d'un producteur (" + i + ").")
            
            repeat 
            {
                sleep(500)
                message = "Objet numero " + nb
                out ! message
                println("Production " + i + " : " + message + "...>>")
                nb = nb + 1
            }
        }
    }
    
\end{lstlisting}

\begin{lstlisting}[frame=trBL,title={Producteurs-Consommateurs: Consommateur}, firstnumber=last]
    class Consommateur(i: Int, in: ?[String]) {
        def consommer : PROC = {
            println("Creation d'un consommateur (" + i + ").")
            var message =""
            repeat
            {
                sleep(500)
                var message =""
                repeat 
                    {
                    message=in ?; 
                    println("<<Consommation " + i + " : " + message)
                }
             }
        }
    } 
     
\end{lstlisting}

\begin{lstlisting}[frame=trBL,title={Producteurs-Consommateurs : Buffer}, firstnumber=last]    
    class Buffer( in: ?[String], out: ![String]) {
        def process : PROC = {
            println("Entree Buffer")
            var message =""
            repeat 
            {
                message=in ?; 
                out!message
            }
        }
    }
\end{lstlisting} 

L'implémentation de cet algorithme est réalisé d'une manière assez simple, avec seulement un pipe vers le buffer qui redirige vers n'importe quel consommateur.\\
Le résultat obtenu est conforme à celui attendu :

\begin{lstlisting}[frame=trBL,title={Producteurs-Consommateurs : Résultat de l'éxécution}, firstnumber=last]   
     [exec] Debut du programme Producteur-Consommateur (4,1).
     [exec] Creation d'un producteur (0).
     [exec] Creation d'un producteur (1).
     [exec] Creation d'un producteur (2).
     [exec] Creation d'un producteur (3).
     [exec] Creation d'un consommateur (0).
     [exec] Entree Buffer
     [exec] Production 3 : Objet numero 0...>>
     [exec] <<Consommation 0 : Objet numero 0
     [exec] <<Consommation 0 : Objet numero 0
     [exec] <<Consommation 0 : Objet numero 0
     [exec] Production 0 : Objet numero 0...>>
     [exec] <<Consommation 0 : Objet numero 0
     [exec] Production 1 : Objet numero 0...>>
     [exec] Production 2 : Objet numero 0...>>
     [exec] Production 3 : Objet numero 1...>>
     [exec] <<Consommation 0 : Objet numero 1
     [exec] Production 0 : Objet numero 1...>>
     [exec] Production 1 : Objet numero 1...>>
     [exec] <<Consommation 0 : Objet numero 1
     [exec] <<Consommation 0 : Objet numero 1
     [exec] Production 2 : Objet numero 1...>>
     [exec] <<Consommation 0 : Objet numero 1
     [exec] <<Consommation 0 : Objet numero 2
     [exec] Production 3 : Objet numero 2...>>
     [exec] Production 0 : Objet numero 2...>>
     [exec] Production 1 : Objet numero 2...>>
     [exec] <<Consommation 0 : Objet numero 2
     [exec] <<Consommation 0 : Objet numero 2
     [exec] <<Consommation 0 : Objet numero 2
     [exec] Production 2 : Objet numero 2...>>
     [exec] Production 3 : Objet numero 3...>>
     [exec] <<Consommation 0 : Objet numero 3
     [exec] Production 0 : Objet numero 3...>>
     [exec] Production 1 : Objet numero 3...>>
     [exec] <<Consommation 0 : Objet numero 3
     [exec] <<Consommation 0 : Objet numero 3
     [exec] Production 2 : Objet numero 3...>>
     [exec] <<Consommation 0 : Objet numero 3
     [exec] <<Consommation 0 : Objet numero 4
     [exec] Production 3 : Objet numero 4...>>
     [exec] <<Consommation 0 : Objet numero 4
     [exec] Production 1 : Objet numero 4...>>
     [exec] <<Consommation 0 : Objet numero 4
     [exec] Production 0 : Objet numero 4...>>
     [exec] Production 2 : Objet numero 4...>>
     [exec] <<Consommation 0 : Objet numero 4
     [exec] Production 3 : Objet numero 5...>>
     [exec] <<Consommation 0 : Objet numero 5
     [exec] Production 1 : Objet numero 5...>>
     [exec] <<Consommation 0 : Objet numero 5
     [exec] Production 0 : Objet numero 5...>>
     [exec] <<Consommation 0 : Objet numero 5
     [exec] Production 2 : Objet numero 5...>>
     [exec] <<Consommation 0 : Objet numero 5
\end{lstlisting} 

\chapter[JCSP]{Java Communicating Sequential Processes -- JCSP}

JCSP est une implémentation de CSP pour le langage de programmation JAVA. Il fait partie des solutions les plus populaires du fait de sa simplicité et de la communauté active de développement et de forums. Développé à l'université de Kent par Peter Welch et Paul Austin, il dispose de plus d'une documentation détaillée.

Nous allons montrer quelques possibilités offertes par ce langage puis le résultat d'implémentation des problèmes classiques de producteur/consommateur et du dîner des philosophes.

\section{Utilisation de JCSP}

\section{Producteur/Consommateur}

L'implémentation du problème Producteur/Consommateur est relativement aisé en JCSP. On crée trois classes : Producteur, Consommateur et Main. Voici les programmes détaillés :

\begin{lstlisting}[frame=trBL,title={Producteurs-Consommateurs: Producteur.java}]
import org.jcsp.lang.*;
public class Producteur implements CSProcess
{
  final private ChannelOutput out;
  public Producteur (ChannelOutput out)
        {
                this.out = out ;
        }
        public void run ()
        {
                for (int i=1;i<=100 ; i=i+1)
                {
                        out.write (i);
                }
        }
}
\end{lstlisting}

\begin{lstlisting}[frame=trBL,title={Producteurs-Consommateurs: Consommateur.java}]
import org.jcsp.lang.*;
public class Consommateur implements CSProcess
{
        final private ChannelInput in;
        public Consommateur (ChannelInput in)
        {
                this.in = in;
        }
        public void run ()
        {
                while (true)
                {
			Integer d = (Integer) in.read();
                        System.out.println ("Lit :" +d);
                }
        }
}
\end{lstlisting}

\begin{lstlisting}[frame=trBL,title={Producteurs-Consommateurs: Main.java}]
import org.jcsp.lang.*;

public class Main
{
        public static void main (String[] args)
        {
	final One2OneChannel c = Channel.one2one();
             new Parallel
		(
		new CSProcess[]
			{
			new Producteur(c.out()) ,
			new Consommateur(c.in())
			} 
		).run ();
	}
}
\end{lstlisting}

Voici le résultat de l'execution de ce programme :

Lit :2
Lit :3
Lit :4
Lit :5
Lit :6
Lit :7
Lit :8
Lit :9
Lit :10
Lit :11
Lit :12
Lit :13
Lit :14
Lit :15
Lit :16
Lit :17
Lit :18
Lit :19
Lit :20
Lit :21
Lit :22
Lit :23
Lit :24
Lit :25
Lit :26
Lit :27
Lit :28
Lit :29
Lit :30
Lit :31
Lit :32
...

\section{Dîner des philosophes}

L'implémentation du dîner des philosophes présentée ci-dessous est directement inspirée de celle présente dans les démonstrations et fournie avec JCSP. Voici les programmes esssentiels. Pour les programmes annexes, vous pouvez vous référer aux fichiers présents dans le fichier jar joint à notre projet.

\begin{lstlisting}[frame=trBL,title={Dîner des philosophes : DeadMain.java}]
import org.jcsp.lang.*;
import org.jcsp.demos.util.*;

public class DeadMain {


  public static void main (String[] args) {

  	Ask.app (TITLE, DESCR);
  	Ask.addPrompt ("Nombre de philosophes", 1, 100, 5);
  	Ask.show ();
  	final int nPhilosophers = Ask.readInt ("Nombre de philosophes");
  	Ask.blank ();

    Any2OneChannel report = Channel.any2one ();

    new Parallel (
      new CSProcess[] {
        new DiningPhilosophersCollege (nPhilosophers, report.out ()),
        new TextDisplay (nPhilosophers, report.in ())
      }
    ).run ();
  }
}
\end{lstlisting}

\begin{lstlisting}[frame=trBL,title={Dîner des philosophes : DiningPhilosophersCollege.java}]
import org.jcsp.lang.*;

class DiningPhilosophersCollege implements CSProcess {

  private final int nPhilosophers;
  private final ChannelOutput report;

  public DiningPhilosophersCollege (int nPhilosophers, ChannelOutput report) {
    this.nPhilosophers = nPhilosophers;
    this.report = report;
  }

  public void run () {

    final One2OneChannel[] left = Channel.one2oneArray (nPhilosophers);
    final One2OneChannel[] right = Channel.one2oneArray (nPhilosophers);

    final Fork[] fork = new Fork[nPhilosophers];
    for (int i = 0; i < nPhilosophers; i++) {
      fork[i] = new Fork (nPhilosophers, i,
                          left[i].in (), right[(i + 1)%nPhilosophers].in (), report);
    }

    final Philosopher[] phil = new Philosopher[nPhilosophers];
    for (int i = 0; i < nPhilosophers; i++) {
      phil[i] = new Philosopher (i, left[i].out (), right[i].out (), report);
    }

    new Parallel (
      new CSProcess[] {
        new Parallel (phil),
        new Parallel (fork),
        new Clock (report)
      }
    ).run ();
  }
}
\end{lstlisting}

\begin{lstlisting}[frame=trBL,title={Dîner des philosophes : Fork.java}]
import org.jcsp.lang.*;
class Fork implements CSProcess {

  protected int id;
  protected Integer Id;
  protected AltingChannelInput left, right;
  protected ChannelOutput report;

  protected ForkReport leftUp, leftDown, rightUp, rightDown;

  public Fork (int nPhilosophers, int id,
               AltingChannelInput left, AltingChannelInput right,
               ChannelOutput report) {
    this.id = id;
    Id = new Integer(id);
    this.left = left;
    this.right = right;
    this.report = report;
    leftUp = new ForkReport (id, id, ForkReport.UP);
    leftDown = new ForkReport (id, id, ForkReport.DOWN);
    rightUp = new ForkReport ((id + 1) % nPhilosophers, id, ForkReport.UP);
    rightDown = new ForkReport ((id + 1) % nPhilosophers, id, ForkReport.DOWN);
  }

  public void run () {
    Alternative alt = new Alternative (new Guard[] {left, right});
    final int LEFT = 0;
    final int RIGHT = 1;
    while (true) {
      switch (alt.fairSelect ()) {
        case LEFT:
          left.read ();
          report.write (leftUp);
          left.read ();
          report.write (leftDown);
        break;
        case RIGHT:
          right.read ();
          report.write (rightUp);
          right.read ();
          report.write (rightDown);
        break;
      }
    }
  }
}
\end{lstlisting}

\begin{lstlisting}[frame=trBL,title={Dîner des philosophes : Philosopher.java}]
import java.util.Random;
import org.jcsp.lang.*;
import org.jcsp.plugNplay.*;

class Philosopher implements CSProcess {

  protected final static int seconds = 1000;

  protected final static int maxThink = 3*seconds;
  protected final static int maxEat = 5*seconds;

  protected final static String[] space =
    {"  ", "    ", "      ", "        ", "          "};

  protected final int id;
  protected final ChannelOutput left, right;
  protected final ChannelOutput report;

  protected final Random random;

  // Constructeurs

  public Philosopher (int id, ChannelOutput left, ChannelOutput right, ChannelOutput report) {
    this.id = id;
    this.left = left;
    this.right = right;
    this.report = report;
    this.random = new Random (id + 1);
  }

  public void run () {

    final Integer Id = new Integer (id);
    final CSTimer tim = new CSTimer ();

    final PhilReport thinking = new PhilReport (id, PhilReport.THINKING);
    final PhilReport hungry = new PhilReport (id, PhilReport.HUNGRY);
    final PhilReport sitting = new PhilReport (id, PhilReport.SITTING);
    final PhilReport eating= new PhilReport (id, PhilReport.EATING);
    final PhilReport leaving = new PhilReport (id, PhilReport.LEAVING);

    final ProcessWrite signalLeft = new ProcessWrite (left);
    signalLeft.value = Id;

    final ProcessWrite signalRight = new ProcessWrite (right);
    signalRight.value = Id;

    //final CSProcess signalForks = new Parallel (new CSProcess[] {signalLeft, signalRight});
    final CSProcess signalForks = new Sequence (
    	new CSProcess[] {
    		signalLeft,
    		new CSProcess () { public void run () { tim.sleep (seconds); } },
    		signalRight
    	}
    );

    while (true) {
      report.write (thinking);
      tim.sleep (range (maxThink));    // pense
      report.write (hungry);
      report.write (sitting);
      signalForks.run ();              // pick up my forks (in parallel)
      report.write (eating);
      tim.sleep (range (maxEat));      // mange
      report.write (leaving);
      signalForks.run ();              // put down my forks (in parallel)
    }
  }

  protected int range (int n) {
    // returns random int in the range 0 .. (n - 1)  [This is not needed in JDK 1.2.x]
    int i = random.nextInt ();
    if (i < 0) {
      if (i == Integer.MIN_VALUE) {
        i = 42;
      } else {
        i = -i;
      }
    }
    return i % n;
  }
}
\end{lstlisting}

\chapter{Conclusion}

Ce projet d'application nous a permis de prendre en main différentes technologies que nous n'avions pas obligatoirement vues en cours : CSO, Scala, JCSP. De plus, nous avons pu mettre en pratique, les différents éléments vus en cours de Parallélisme et Systèmes Répartis et de les approfondir. En bilan, nous avons étudié les différents mécanismes en jeu pour la programmation concurrente pour les langages Scala et Java. Nous avons programmé les principaux problèmes classiques dans ces bibliotèques. De surcroît, nous fournissons des tutoriels permettant de faciliter leur utilisation. L'objectif étant de les utiliser pour les promotions suivantes à la manière de CTJ actuellement.

\appendix
\chapter{ Tour d'horizon de CSO }
\label{chap:cso}
\section{Généralités sur Scala}
\begin{description}
\item[Déclaration de variable] La déclaration d'un champ se présente comme suit:
\begin{verbatim}
[visibilité] (var|val) nom [:Type] [= Valeur]
\end{verbatim}
\item[Déclaration de classes] Les classes se manipulent avec beaucoup plus de flexibilité que les autres langages de programmation orienté objet, comme le Java. Par exemple, la déclaration d'un enfant par son nom et le nombre de billes qu'il possède s'écrira :
\begin{lstlisting}[frame=trBL]
class Enfant {
	var nom : String = _
	private var billes = _ 
} 
\end{lstlisting}
On remarque ici que le caractère $\_$ permet d'assigner la valeur adaptée au type du champ : 0 pour les \textsf{Int}, "" pour les \textsf{String}, etc.\\
L'accès aux arguments de la classe se font simplement par 
\begin{lstlisting}[frame=trBL]
object MonExemple {
	def main(args : Array[String]) :{
		val toto = new Enfant
		tot.nom="Toto"
		println(toto.nom)
		toto.billes = toto.billes + 5
	}
} 
\end{lstlisting}

Scala alloue un espace mémoire pour un champ, et génère un getter et un setter qui seront automatiquement invoqués lors de la lecture et écriture des instances. Plus de temps perdu comme en Java, Scala s'occupe de tout.

\item[Quelques mots-clefs] Effectuons un tour d'horizon des mots-clefs indispensables à la compréhension des programmes Scala:

\begin{itemize}
\item[\textsf{object}] En scala, on manipule non seulement des instances des classes, mais aussi d'une manière plus générale des objets. Définir un objet via le mot-clef \textsf{object}, c'est tracer les contours d'une entité que l'on pourra utiliser d'une manière extensible. Nous avons aperçu son utilisation dans l'exemple ci-dessus.\\
\item[\textsf{class}] Utilisation similaire à celle des langages objets. On manipule uniquement des instances de classes.\\

\item[\textsf{var}] définit une variable en tant que telle, c'est-à-dire qui pourra changer au fil du programme \textit{i.e.} dont la valeur est modifiable. Pour ajouter par exemple un élément au setter du nombre de billes de l'enfant (\emph{cf.} ci-dessus) :
\begin{lstlisting}[frame=trBL]
class Enfant(var name : String, private var b: Int) {
	def billes:String = b + " billes"
	def billes_=(b: Int) { if (b<0) error("Nombre de billes negatif") else this.b = b}
}
\end{lstlisting}
On a ici d'abord rajouter \textsf{private} à la déclaration de \textsf{billes} dans le constructeur (en le renommant par la même occasion en \textsf{b}), et définit deux méthodes (via le mot-clef \textsf{def}) :

\begin{itemize}
\item billes : qui retourne en \textsf{String} le nombre de billes du champ privé en lui rajoutant la cha\^ine \textsf{" billes"}.
\item billes\_ : qui est bien un identificateur valide en Scala et qui teste la positivité du nombre de billes; de telle sorte que \textsf{toto.billes = -5} lève une erreur comme prévu.\\
\end{itemize}

\item[\textsf{val}] définit une constante; toute modification lèvera une exception.\\

\item[\textsf{def}] permet de définir méthodes, variables, et bien plus encore. Par exemple \textsf{def pp = \{ v => println(v)\} } définit une fonction qui affiche l'argument d'entrée dans la console.
\end{itemize}
\end{description}

\section{Communcating Scala Objects}
\begin{description}
\item[Processus] Un processus en CSO est une extension du PROC de Scala qui permet de lancer un thread. D'après le document de Dr.\ \textsc{Suffrin} \cite{cpa2008-cso}, on a:
\begin{center}
\begin{tabular}{lp{0.6\textwidth}}
\textsc{PROC} \{ \emph{expr}\} & Un simple processus (\emph{expr} doit être une commande, c'st-à-dire du type \textsf{Unit})\\ 
\emph{$p_{1}$} || \emph{$p_{2}$} || ... || \emph{$p_{n}$}  & Composition parallèle de \emph{n} processus (chaque \emph{$p_{i}$} doivent être du type \textsc{PROC})\\
|| \emph{collection} & Composition parallèle pour une collection finie de différents \textsc{PROC}. \\ 
\end{tabular} 
\end{center}
Une écriture récurrente dans nos programmes est \textsf{||(for ( i <-0 until n) yield $p_{i}$ )}, et permet de lancer un nombre fini de processus en parallèle. Elle est équivalente à \textsf{ $p_{0}$ ||  $p_{1}$ ||  $p_{n-1}$ }.
Pour lancer un \textsc{PROC}, il suffit d'ajouter deux parenthèses après \textsf{$p_i()$}

\begin{lstlisting}[frame=trBL]
( 
  proc { val i = 0 ; repeat { println("tic " + i) ; i = i + 1 } }
|| 
  proc { repeat { println(println("tac " + i) ; i = i + 1) } }
)()
\end{lstlisting}

\item[Channel] Les \textsf{channels}, tels des pipes, synchronisent la terminaison de l'instruction \textsf{!} à la sortie du port, avec la terminaison de l'instruction \textsf{?} correspondante à son entrée :

\begin{center}
\begin{tabular}{lp{0.6\textwidth}}
\textsf{OneOne[T]} & Un seul et unique process peut accéder à la fois au port de sortie et d'entrée.\\

\textsf{ManyOne[T]} & Un seul et unique process peut accéder au port d'entrée. Les process qui souhaitent accéder à sa sortie en ont la possibilité dans un ordre aléatoire.\\

\textsf{OneMany[T]} & Un seul et unique process peut accéder au port de sortie. Les process qui souhaitent accéder à son entrée en ont la possibilité dans un ordre aléatoire.\\

\textsf{ManyMany[T]} & Autant de process peuvent accéder à la fois au port de sortie et d'entrée. L'écriture et la lecture se font dans un ordre aléatoire.\\
\end{tabular}
\end{center}

\item[Rendez-vous] Pour une plus grande documentation, se reporter au rapport du Docteur~\textsc{Suffrin} \cite{cpa2008-cso}.\\ 
Pour récupérer la valeur d'un \textsf{channel}, on utilise la syntaxe suivante :
\begin{verbatim}
var message=in ?; 
println("\"" + message + "\" recu.")
\end{verbatim}

Une notation simplifiée est :
\begin{verbatim}
in ? { message => { println( message + "recu.") }  }
\end{verbatim}             

\end{description}


\chapter[Tutoriel JCSP]{Tutoriel -- JCSP}

Voici un tutoriel concernant l'utilisation de JCSP. Tout d'abord, le lecteur doit installer JCSP. Pour cela, il a à disposition le script bash $install_jcsp.sh$. Pour l'exécuter, il suffit de taper dans un terminal :

\begin{lstlisting}[frame=trBL]
bash install_jcsp.sh
\end{lstlisting}

Ce script télécharge les fichiers nécessaires et crée un dossier dans le répertoire personnel nommé JCSP. \`A présent pour utiliser les programmes  fournis un makefile a été crée. Il spécifie les chemins absolus des fichiers jar nécessaires à la compilation du programme. Le lecteur prendra soin à modifier le chemin en fonction du nom de son répertoire. Puis il exécutera successivement :

\begin{lstlisting}[frame=trBL]
make compile
make run
\end{lstlisting}

Les programmes sont alors exécutés. Le fichier principal est par défaut appelé Main.java. Cependant, le lecteur est libre de modifier ce choix. Il suffit de modifier la ligne correspondante dans le makefile.

\begin{thebibliography}{Cpa2008}
\bibitem[Cpa2008]{cpa2008-cso} {\textit{Communicating Process Architectures 2008}, Susan Stepney, Fiona Polack, Alistair McEwan, Peter Welch, and Wilson Ifill (Eds.), IOS Press, 2008
\copyright 2008 The authors and IOS Press. All rights reserved.\\
\url{http://citeseerx.ist.psu.edu/viewdoc/download?doi=10.1.1.164.8856&rep=rep1&type=pdf}}
\end{thebibliography}

\end{document}

