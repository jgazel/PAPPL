\documentclass[a4paper,11pt]{article}
 \usepackage[utf8]{inputenc} %latin9 ou1
 \usepackage[T1]{fontenc}
 \usepackage[normalem]{ulem}
 \usepackage[french]{babel}
 \usepackage{verbatim}
 \usepackage{graphicx}
 \usepackage{multicol}
%\usepackage[letterpaper]{geometry}
%\geometry{verbose,tmargin=3cm,bmargin=3cm,lmargin=3cm}
%\usepackage{varioref}

%\usepackage{babel}
\usepackage{fancyhdr}
\cfoot{Page \thepage}
\pagestyle{fancy}

\usepackage{listings}   % need for code encapsulation
\lstset{
	language=Python,
	numbers=left, numberstyle=\tiny, stepnumber=1, numbersep=7pt,
	keywordstyle=\bfseries\emph,
	breaklines=true,
	frameround=fttt,
	basicstyle= \mdseries\scriptsize }
\lstset{language=bash,basicstyle=\scriptsize,commentstyle=\small\itshape,stringstyle=\ttfamily,numbers=left,numberstyle=\tiny,stepnumber=1,showstringspaces=false}
	
% Commandes personnelles
\def\clap#1{\hbox to 0pt{\hss #1\hss}} % Une commande sembleble  \rlap ou \llap, mais centrant son argument
\def\ligne#1{\hbox to \hsize{\vbox{\centering #1}}} % Une commande centrant son contenu ( utiliser en mode vertical)
% Une comande qui met son premier argument  gauche, le second au 
% milieu et le dernier  droite, la premire ligne ce chacune de ces
% trois boites coincidant
\def\haut#1#2#3{\hbox to \hsize{\rlap{\vtop{\raggedright #1}}\hss \clap{\vtop{\centering #2}} \hss \llap{\vtop{\raggedleft #3}}}}%
% Idem, mais cette fois-ci, c'est la dernire ligne
\def\bas#1#2#3{\hbox to \hsize{\rlap{\vbox{\raggedright #1}} \hss \clap{\vbox{\centering #2}} \hss \llap{\vbox{\raggedleft #3}}}}%
    
    
% La commande \maketitle
\makeatletter
	\def\maketitle{%
	  \thispagestyle{empty}\vbox to \vsize{%
		\vspace{3cm} \ligne{\Huge \@title}
		\vspace{1cm} \haut{Supervisé par \@supervisor}{}{\@follower}
		\vspace{3mm}\hrule
		\vfill
		\haut{}{Jérôme \textsc{Gazel}}{}
		\haut{}{$\&$}{}
		\haut{}{Clément \textsc{Schiano de Colella}}{}
		\vspace{3cm}
		\bas{}{\@location, \@date}{}
		}%
	  %\cleardoublepage
	  }

	% Les commandes permettant de dfinir la date, le lieu, etc.
	\def\date#1{\def\@date{#1}}
	\def\title#1{\def\@title{#1}}
	\def\location#1{\def\@location{#1}}
	\def\blurb#1{\def\@blurb{#1}}
	\def\supervisor#1{\def\@supervisor{#1}}
	\def\follower#1{\def\@follower{#1}}
	\def\email#1{\def\@email{\small{#1}}}
	% Valeurs par dfaut
	\date{\today}
\makeatother


  \title{Compte-Rendu d'Avancement $\no$4}
  \location{Nantes}
  \lhead{\itshape RA $\no$4 [25/02/11]}
  \supervisor{M.\ Olivier \textsc{Roux}}
  \follower{\'Ecole Centrale de Nantes}


\begin{document}
\maketitle

\part*{Avancées }

\section{Problème sur CSO}

Comme nous l'avions indiqué à notre encadrant M.\ \textsc{Roux}, nous avons éprouvé au cours des dernières semaines de grandes difficultés à appréhender CSO -- Communcating Scala Objects.\\

Après avoir construit un fichier \textsf{buid.xml} capable de compiler, nettoyer, lancer le programme en adjoignant les bonnes ressources via une simple commande, puis avoir saisi la syntaxe nécessaire à la programmation concurrentielle, la programmation des deux schémas-types : producteur-consommateur et lecteur-écrivain, a été commencé.\\

Cependant, après des résultats infructueux, il semblerait que notre équipe ait mis à jour un problème relevant du comportement de CSO vis-à-vis des processus concurrents. En effet, leur ordre d'arrivée dépend de leur position dans la sélective; ce qui ne devrait pas être le cas.\\

\subsection{Illustration du problème}

Pour illustrer ces propos, considérons le programme suivant \textsf{HelloWorld} qui lancent plusieurs procesus en parallèle, chacun ayant soit la tâche d'afficher "Hello" ou "World" :
\medskip

\begin{lstlisting}[frame=trBL]
package helloworld
import ox.CSO._

object Hello_Par {

    val NP = 3
    val NC = 4
    
    def main( args: Array[String] ) {
  
        println("Debut du programme HelloWord Par :")
        ( 
           || ( for (i <- 0 until NP) yield Hello(i) )
        || 
           || ( for (i <- 0 until NC) yield World(i) )
        )()
        
    }


    def Hello( i: Int ) : PROC={
    	say("Hello" + i)
    } 

    def World( i: Int ) : PROC={
    	say("World" + i)
    }

    def say( word: String ) {
      println(word)
    }  
}
\end{lstlisting}

Les résultats obtenus avec cette syntaxe sont les suivants :
\medskip
\begin{multicols}{2}

\begin{lstlisting}[frame=trBL]
     [exec] Debut :
     [exec] Hello0
     [exec] Hello1
     [exec] World3
     [exec] World2
     [exec] World0
     [exec] World1
     [exec] Hello2
     [exec] Debut :
     [exec] Hello2
     [exec] Hello1
     [exec] Hello0
     [exec] World3
     [exec] World2
     [exec] World0
     [exec] World1
     [exec] Debut  :
     [exec] Hello2
     [exec] Hello0
     [exec] Hello1
     [exec] World3
     [exec] World0
     [exec] World2
     [exec] World1
     [exec] Debut  :
     [exec] Hello2
     [exec] Hello0
     [exec] Hello1
     [exec] World3
     [exec] World2
     [exec] World1
     [exec] World0
     [exec] Debut  :
     [exec] Hello2
     [exec] Hello1
     [exec] Hello0
     [exec] World3
     [exec] World2
     [exec] World0
     [exec] World1
     [exec] Debut  :
     [exec] Hello2
     [exec] Hello1
     [exec] Hello0
     [exec] World3
     [exec] World2
     [exec] World0
     [exec] World1
     [exec] Debut  :
     [exec] Hello2
     [exec] Hello0
     [exec] Hello1
     [exec] World3
     [exec] World1
     [exec] World0
     [exec] World2
     [exec] Debut  :
     [exec] Hello2
     [exec] Hello0
     [exec] Hello1
     [exec] World3
     [exec] World2
     [exec] World1
     [exec] World0
     [exec] Debut  :
     [exec] Hello2
     [exec] World2
     [exec] Hello1
     [exec] Hello0
     [exec] World3
     [exec] World0
     [exec] World1
     [exec] Debut  :
     [exec] Hello2
     [exec] Hello0
     [exec] Hello1
     [exec] World3
     [exec] World2
     [exec] World0
     [exec] World1
     [exec] Debut  :
     [exec] Hello2
     [exec] Hello1
     [exec] Hello0
     [exec] World3
     [exec] World2
     [exec] World0
     [exec] World1
     [exec] Debut  :
     [exec] Hello2
     [exec] Hello1
     [exec] Hello0
     [exec] World3
     [exec] World2
     [exec] World0
     [exec] World1
     [exec] Debut  :
     [exec] Hello2
     [exec] Hello0
     [exec] Hello1
     [exec] World3
     [exec] World2
     [exec] World0
     [exec] World1
     [exec] Debut  :
     [exec] Hello2
     [exec] World2
     [exec] World0
     [exec] World1
     [exec] World3
     [exec] Hello0
     [exec] Hello1
     [exec] Debut  :
     [exec] Hello2
     [exec] Hello1
     [exec] Hello0
     [exec] World3
     [exec] World2
     [exec] World0
     [exec] World1
     [exec] Debut  :
     [exec] Hello2
     [exec] Hello0
     [exec] Hello1
     [exec] World3
     [exec] World2
     [exec] World0
     [exec] World1
     [exec] Debut  :
     [exec] Hello2
     [exec] Hello1
     [exec] Hello0
     [exec] World3
     [exec] World2
     [exec] World0
     [exec] World1

\end{lstlisting}
\end{multicols}


On remarque bien que les processus dépendent de leur place dans la sélective, et leur ordre est même (sauf exceptions) quasiment le même.

\subsection{Bilan}

Malgré ce problème, nous avons rédigé un programme de producteur-consommateur en CSO. Une mauvaise communication entre les différents processus sustistent encore. Nous continuons notre travail en nous penchant à deux sur ces deux sujets.
\end{document}

