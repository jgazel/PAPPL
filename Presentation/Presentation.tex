\documentclass[slidetop,11pt]{beamer}
%
% Ces deux lignes à décommenter pour sortir 
% le texte en classe article
% \documentclass[class=article,11pt,a4paper]{beamer}
% \usepackage{beamerbasearticle}

% Packages pour les français
%
\usepackage[T1]{fontenc} 
\usepackage[utf8]{inputenc}
\usepackage[frenchb]{babel}
% pour un pdf lisible  l'cran si on ne dispose pas 
% des fontes cmsuper ou lmodern
\usepackage{lmodern}
%\usepackage{aeguill}

% Pour afficher le pdf en plein ecran
% (comment pour imprimer les transparents et pour les tests)
%\hypersetup{pdfpagemode=FullScreen}

\usepackage{listings}   % need for code encapsulation

\lstset{
	language=Python,
	numbers=left, numberstyle=\tiny, stepnumber=1, numbersep=7pt,
	morekeywords={%
    %%% BOUCLE, TEST & Co.
      if, elif, then, else, and, or, until, yield,repeat,
    %%% IMPORT & Co.
      import, from, package,
    %%% FONCTIONS NUMERIQUES
      cos, sin, tan, acos, asin, atan,
    %%% CONSTANTES
      pi, True, False,
    %%% Types .
      Int, String, Array, Integer, 
    %%% Mots-Clefs supp.
      object, class, protected, public, private, final, val, var, new, ?, !
    },
    sensitive=true,
    morecomment=[l]{\#},
	keywordstyle=\bfseries\emph,
	breaklines=true,	
	frameround=fttt,
	basicstyle= \mdseries\scriptsize }
%\lstset{language=bash,basicstyle=\scriptsize,commentstyle=\small\itshape,stringstyle=\ttfamily,numbers=left,numberstyle=\tiny,stepnumber=1,showstringspaces=false}
	

% ------------------------------------------------
%-----------   styles pour beamer   --------------
% ------------------------------------------------
%
% ------------- Choix des couleurs ---------------
%\xdefinecolor{fondtitre}{rgb}{0.20,0.43,0.09}  % vert fonce
% la mme d'une autre manire
\definecolor{fondtitre}{HTML}{336E17}

%\xdefinecolor{coultitre}{rgb}{0.41,0.05,0.05}  % marron
% la mme d'une autre manire
\definecolor{coultitre}{RGB}{105,13,13}

\xdefinecolor{fondtexte}{rgb}{1,0.95,0.86}     % ivoire

% Redfinit la couleur de fond pour imprimer sur transparents
%\xdefinecolor{fondtexte}{rgb}{1,1,1}     % blanc

% commande differente pour les couleurs nommes - de base
%\colorlet{coultexte}{black} 

% -------------- Fioritures de style -------------
% Fait afficher l'ensemble du frame 
% en peu lisible (gris clair) ds l'ouverture
\beamertemplatetransparentcovered

% Supprimer les icones de navigation (pour les transparents)
%\setbeamertemplate{navigation symbols}{}

% Mettre les icones de navigation en mode vertical (pour projection)
%\setbeamertemplate{navigation symbols}[vertical]

% ------------ Choix des thmes ------------------
%\usecolortheme{crane}
%\usetheme{darmstadt}
%\usetheme{frankfurt}
%\useoutertheme{default}
%\useinnertheme{default}
%\useinnertheme[shadow=true]{rounded}
\mode<presentation>
\useoutertheme{tree}
\usecolortheme{whale}
\usecolortheme{orchid}
\useinnertheme[shadow=true]{rounded}

\setbeamerfont{block title}{size={}}

% Dfinition de boites en couleur spcifiques
% premire mthode
\setbeamercolor{bas}{fg=coultitre, bg=fondtitre!40}
\setbeamercolor{haut}{fg=fondtitre!40, bg=coultitre}
% deuxime mthode
\beamerboxesdeclarecolorscheme{clair}{fondtitre!70}{coultitre!20}
\beamerboxesdeclarecolorscheme{compar}{coultitre!70}{fondtitre!20}

% insrer le nombre de pages
%\logo{\insertframenumber/\inserttotalframenumber}

%------------ fin style beamer -------------------

% Faire apparatre un sommaire avant chaque section
% \AtBeginSection[]{
%   \begin{frame}
%   \frametitle{Plan}
%   \medskip
%   %%% affiche en dbut de chaque section, les noms de sections et
%   %%% noms de sous-sections de la section en cours.
%   \small \tableofcontents[currentsection, hideothersubsections]
%   \end{frame} 
% }

% ----------- Contenu de la page de titre --------
\title{Programmation Concurrente}
\subtitle{Communicating Scala Objects\\Java Communicating Sequential Processes}
%\subtitle{Fichier test}
\author{Jérôme \textsc{Gazel} \\ Clément \textsc{Schiano de Colella}}
\institute{\'Ecole Centrale de Nantes}
\date{\oldstylenums{vendredi 25 mars 2011}}
\logo{\includegraphics[height=1cm]{Logo_ECN.png}}
% ------------------------------------------------
% -------------   Début document   ---------------
% ------------------------------------------------
\begin{document}
%--------- Ecriture de la page de titre ----------
% avec la commande frame simplifie
\frame{\titlepage}
%
\part{Projet d'Application} 
%------------------- Sommaire ----------------
\begin{frame}{Sommaire}
  \small \tableofcontents[hideallsubsections]
\end{frame} 


% ---------------------------------------------
% ------------ INTRODUCTION-------------
% ---------------------------------------------
\section{Introduction}
% Sommaire local. En deux colonnes
\begin{frame}{Plan}
  \tableofcontents[sections=\thesection]
\end{frame}
% ---------------------------------------------
\subsection{Programmation concurrente}
% avec l'environnement frame
\begin{frame}[label=pagesimple]
  \frametitle{Programmation concurrente}
    \begin{itemize}[<+->]
  \item Qu'est-ce que la programmation concurrente?
  \item Quelles sont ces intérêts?
   \item Quels sont les outils disponibles?
    \end{itemize}
\end{frame}

\begin{frame} 
  \frametitle{Présentation de CSP}
  Article célèbre de Hoare
  CSP est un langage :
\begin{itemize}[<+->]
\item impératif
\item parallèle
\item communications par échanges synchrones de message
 \end{itemize}
\end{frame}

\begin{frame} [containsverbatim]
\framesubtitle{Notion de rendez-vous}
Exemple d'échange de message :
\begin{lstlisting}[frame=trBL]
P:: Q ! M(x)
Q:: P ? M(y)
\end{lstlisting}
\end{frame}

\begin{frame} [containsverbatim]
\framesubtitle{Notion de garde et commande gardée}

 \begin{lstlisting}[frame=trBL]
<commande gardee> ::= <garde> -> <suite d instructions>
<garde> ::= <CB>, <CES>
\end{lstlisting}

\begin{lstlisting}[frame=trBL,title={Sélective}]
[[
(state=1) -> state:=state+1;
[]
(state=1) -> state:=state-1;
]]
\end{lstlisting}

\begin{lstlisting}[frame=trBL,title={Répétitive}]
x:=1;
*[[
(x<=N) -> imprimer(x); x++;
[]
(x=N) -> skip;
]]
\end{lstlisting}
\end{frame}


\begin{frame} 
  \frametitle{Conseils à suivre}
  \framesubtitle{Des gestes quotidiens}
  \begin{block}{Nos conseils}
     \begin{itemize}[<+->]
       \item Même en veille, un appareil électrique consomme de l'énergie.
       \item Préférez un ordinateur portable à un ordinateur de Bureau.
       \item Éteignez votre modem / box Internet la nuit.
     \end{itemize}
  \end{block}   
\end{frame}



\section{Bibliographie}

\begin{frame} 
\frametitle{Bibliographie}

\begin{thebibliography}{Cpa2008}
\bibitem[Pasyr]{pasyr} {\textit{Cours de Parallélisme et Systèmes Répartis}, Olivier Roux, \'Ecole Centrale de Nantes, 2011}

\bibitem[Cpa2008]{cpa2008-cso} {\textit{Communicating Process Architectures 2008}, Susan Stepney, Fiona Polack, Alistair McEwan, Peter Welch, and Wilson Ifill (Eds.), IOS Press, 2008
\copyright 2008 The authors and IOS Press. All rights reserved.\\
\url{http://citeseerx.ist.psu.edu/viewdoc/download?doi=10.1.1.164.8856&rep=rep1&type=pdf}}
\end{thebibliography}
\end{frame}

\end{document}
